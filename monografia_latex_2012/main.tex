% Document type and layout.
\documentclass[12pt,a4paper,espaco=umemeio,noindentfirst,oneside,openany,tocpage=plain,pnumromarab,ruledheader,time,anapcustomindent]{abnt}

% Brazilian portuguese language stuff (acents, etc.).
\usepackage[brazil]{babel}

% Encoding (or charset).
% This is a feature depending on the configuration of the text editor
% or the particular system which you are using.
\usepackage[utf8]{inputenc}
% If you experience some problem with the charset encoding (weird
% simbols being displayed) try to change the option utf8 below to
% latin1 (which sets the charset to ISO-8859-1 encoding).
%\usepackage[latin1]{inputenc}

% Load the package abncite, for bibliographical citations into the
% text body. The option alf below sets the layout of citations to
% Autor-Date style.
%\usepackage[alf,recuo=0.5cm,abnt-emphasize=bf,abnt-and-type=&,abnt-etal-list=3,abnt-etal-cite=3]{abntcite}
% If you want to use numbers, change alf to num, or comment out the
% line below.
%\usepackage[num,recuo=0.5cm,abnt-emphasize=bf,abnt-and-type=&,abnt-etal-list=3,abnt-etal-cite=3]{abntcite}

%%%%%%%%%%%%%%%%%%%%%%%%%%%%%%%%%%%%%%%%%%%%%%%%%%%%%%%%%%%%%%%%%%%%%%

\newif\ifpdf
\ifx\pdfoutput\undefined
	\pdffalse
\else
	\pdftrue
\fi

\ifpdf
\pdfoutput=1
\usepackage[pdftex]{graphicx}
\usepackage[output=pdf]{sty/logo-each}
\else
% Enables the usage of graphics in .jpg format, rather than .eps
% (Encapsulated PostScript).
% Furthermore, is also required by package logo-each below.
\usepackage[dvips]{graphicx}
\usepackage{sty/logo-each}	% EACH-USP logo.
\fi

%%%%%%%%%%%%%%%%%%%%%%%%%%%%%%%%%%%%%%%%%%%%%%%%%%%%%%%%%%%%%%%%%%%%%%

% Additional ABNTex definitions.
\usepackage[disable=copyright]{sty/ach2017}

% This package seems to conflicts with logo-each above, if loaded
% before that package. Therefore, put it here.
\usepackage[T1]{fontenc}

% To type URL with linebreak at special characters.
\usepackage{url}

\usepackage[unicode=true,bookmarks=true,bookmarksnumbered=false,
bookmarksopen=true,bookmarksopenlevel=1,breaklinks=false,
pdfborder={0 0 1},backref=section,colorlinks=false]{hyperref}

% Improved and customizable hyphenation patterns.
\usepackage{hyphenat}
\hyphenation{pe-rio-do res-pon-sá-vel}

%%%%%%%%%%%%%%%%%%%%%%%%%%%%%%%%%%%%%%%%%%%%%%%%%%%%%%%%%%%%%%%%%%%%%%

\renewcommand{\ABNTchapterfont}{\fontfamily{ptm}\bfseries\selectfont}
\renewcommand{\tituloformat}{\huge\ABNTchapterfont}

%%%%%%%%%%%%%%%%%%%%%%%%%%%%%%%%%%%%%%%%%%%%%%%%%%%%%%%%%%%%%%%%%%%%%%

% Define the base path for the graphic files, anchored to the
% directory in which this .sty file is placed (recommended).
%\graphicspath{{./pictures/}}

%%%%%%%%%%%%%%%%%%%%%%%%%%%%%%%%%%%%%%%%%%%%%%%%%%%%%%%%%%%%%%%%%%%%%%

\begin{document}

%%%%%%%%%%%%%%%%%%%%%%%%%%%%%%%%%%%%%%%%%%%%%%%%%%%%%%%%%%%%%%%%%%%%%%
% Fill the fields below, in order to provide the informations for the
% pre-textual elements.

\instituicao{Universidade de São Paulo\\Escola de Artes, Ciências e
  Humanidades}

\autor[Rocha, Larissa Teles da]{Larissa Teles da Rocha}

\titulo{Tripulações aéreas e hotéis: os critérios de escolha
  utilizados por companhias aéreas e as adaptações realizadas pelos
  hoteis em seus produtos e serviços}

\local{São Paulo}

\data{Mês de 2012}

\comentario{Monografia a ser apresentada à Escola de
  Artes, Ciências e Humanidades da Universidade de São Paulo, como
  parte dos requisitos exigidos para aprovação na disciplina
  \disciplinaname, do curso de Bacharelado em Sistemas de Informação.}

%Opte por uma dessas modalidades:
\area[Modalidade:]{TCC Curto (1 semestre) -- individual.}
%\area[Modalidade:]{TCC Curto (1 semestre) -- em grupo.}
%\area[Modalidade:]{TCC Longo (1 ano) -- individual.}
%\area[Modalidade:]{TCC Longo (1 ano) -- em grupo.}

\orientador{Prof. Dr. Renato Braz Oliveira de Seixas} 

\defesadata{28/06/2012}

%Coloque aqui o nome do professor da disciplina  
\bancaone{Profª Drª Fulana} 

% Coloque aqui o nome do parecerista se já souber, caso contrário deixe em branco
\bancatwo{Prof. Dr. Beltrano} 

%Nome do orientador
\adviser[Seixas, Renato Braz Oliveira de]{Renato Braz Oliveira de
  Seixas}

\paginas{74 p. : il.}

\catalogtop{{\ABNTtitulodata} / {\ABNTautordata} ; orientação de
  \ABNTadviserdata. -- {\ABNTlocaldata} : Universidade de São Paulo,
  Escola de Artes, Ciências e Humanidades, \ABNTdatadata.}

\catalogmed{Monografia apresentada para Conclusão de Curso de Lazer e
  Turismo -- Escola de Artes, Ciências e Humanidades da Universidade
  de São Paulo.} 

\catalogbottom{1. Hotelaria. 2. Empresas de transportes
  turísticos. I. \ABNTvaradviserdata, orient. II. Título.}

\catalogcdd{CDD 22.ed. -- 910.46}



%%%%%%%%%%%%%%%%%%%%%%%%%%%%%%%%%%%%%%%%%%%%%%%%%%%%%%%%%%%%%%%%%%%%%%
% Pre-Textual elements without title and without numeric indexation
% (5.3.4).

% Mandatory (4.1.1 Capa).
\capa

% Mandatory (4.1.3 Folha de rosto).
\folhaderosto
\folhaderostoreverso

% Mandatory (4.1.5 Folha de aprovação).
\folhadeaprovacao

%%%%%%%%%%%%%%%%%%%%%%%%%%%%%%%%%%%%%%%%%%%%%%%%%%%%%%%%%%%%%%%%%%%%%%
% Pre-Textual elements without numeric indexation (5.3.3). The title
% of these elements must be centralized, as stated in NBR 6024.

% Optional (4.1.6 Dedicatória).
\pretextualchapter{Dedicatória} % Title.

bla bla bla bla bla bla bla bla bla bla bla bla

bla bla bla bla bla bla bla bla bla bla bla bla

% Optional (4.1.7 Agradecimentos).
\pretextualchapter{Agradecimentos} % Title.

bla bla bla bla bla bla bla bla bla bla bla bla

bla bla bla bla bla bla bla bla bla bla bla bla


\pretextualchapter{Glossário} % Title.

Aqui as palavras aparecerão em ordem alfabética. A palavra ou sigla a ser definida aparecerá em negrito seguida de dois pontos (:), e em seguida a definição é escrita sem negrito. Ex:

{\bf GLCE}: Gramática Livre de Contexto Estocástica – gramática livre de contexto com uma distribuição de probabilidades sobre as produções com o mesmo lado esquerdo.\\

No caso de abreviaturas (siglas), mesmo descrevendo-as aqui não deixe de defini-las na primeira vez em que elas são empregadas!


% Mandatory (4.1.9 Resumo na língua vernácula).
\palavraschave{ABNT, norma NBR 6028, elementos pre-textuais.}
\begin{resumo}

bla bla bla bla bla bla bla bla bla bla bla bla

bla bla bla bla bla bla bla bla bla bla bla bla

\makekeywords
\end{resumo}

% Optional (4.1.11 Lista de ilustrações).
\listadefiguras

% Optional (4.1.11 Lista de tabelas).
\listadetabelas

% Mandatory (4.1.15 Sumário).
\sumario

%%%%%%%%%%%%%%%%%%%%%%%%%%%%%%%%%%%%%%%%%%%%%%%%%%%%%%%%%%%%%%%%%%%%%%
% Textual elements.

\chapter{Introdução} % Chapter title.


bla bla bla bla bla bla bla bla bla bla bla bla

bla bla bla bla bla bla bla bla bla bla bla bla

% ESTA FORMATAÇÃO LARGE É APENAS PARA CHAMAR ATENÇÃO DE VOCÊS SOBRE SEU CONTEÚDO, E NÃO É PARA SER USADA NESTA SEÇÃO.
{\Large Tamanho máximo da monografia final: 30 páginas, contando desde a introdução até a conclusão).

Obs.: No caso de ser um plano para projeto em grupo, deve ficar claro qual o projeto geral, como foi dividido (quantos e quais subprojetos) e qual o subprojeto que cabe ao aluno em questão. Todas as seções daqui em diante são específicas do subprojeto que compete ao aluno.}


%%%%%%%%%%%%%%%%%%%%%%%%%%%%%%%%%%%%%%%%%%%%%%%%%%%%%%%%%%%%%%%%%%%%%%

\chapter{Objetivos}


\section{Objetivo Geral}

Texto no passado!!!!

\section{Objetivos Específicos}


%%%%%%%%%%%%%%%%%%%%%%%%%%%%%%%%%%%%%%%%%%%%%%%%%%%%%%%%%%%%%%%%%%%%%%

\chapter{Revisão bibliográfica}

Exemplo de citação~\cite{Pressman:2006}

%%%%%%%%%%%%%%%%%%%%%%%%%%%%%%%%%%%%%%%%%%%%%%%%%%%%%%%%%%%%%%%%%%%%%%

\chapter{Metodologia}

%%%%%%%%%%%%%%%%%%%%%%%%%%%%%%%%%%%%%%%%%%%%%%%%%%%%%%%%%%%%%%%%%%%%%%

\chapter{Resultados}

%%%%%%%%%%%%%%%%%%%%%%%%%%%%%%%%%%%%%%%%%%%%%%%%%%%%%%%%%%%%%%%%%%%%%%

\chapter{Discussão}

%%%%%%%%%%%%%%%%%%%%%%%%%%%%%%%%%%%%%%%%%%%%%%%%%%%%%%%%%%%%%%%%%%%%%%

\chapter{Conclusão}

bla bla bla bla bla bla bla bla bla bla bla bla

bla bla bla bla bla bla bla bla bla bla bla bla

%%%%%%%%%%%%%%%%%%%%%%%%%%%%%%%%%%%%%%%%%%%%%%%%%%%%%%%%%%%%%%%%%%%%%%
% Post-textual element: Bibliographical references (mandatory).

\bibliographystyle{abnt-alf}

\bibliography{bib.d/example}

%%%%%%%%%%%%%%%%%%%%%%%%%%%%%%%%%%%%%%%%%%%%%%%%%%%%%%%%%%%%%%%%%%%%%%
% Post-textual element: Appendix (optional).

% From now on, the chapters are no longer referenciated by numbers,
% but as "Appendix", and with an alpha-numeric identifier.
\appendix


\chapter{Título do Apêndice A}

bla bla bla bla bla bla bla bla bla bla bla bla

bla bla bla bla bla bla bla bla bla bla bla bla

%%%%%%%%%%%%%%%%%%%%%%%%%%%%%%%%%%%%%%%%%%%%%%%%%%%%%%%%%%%%%%%%%%%%%%

\chapter{Título do Apêndice B}

bla bla bla bla bla bla bla bla bla bla bla bla

bla bla bla bla bla bla bla bla bla bla bla bla

%%%%%%%%%%%%%%%%%%%%%%%%%%%%%%%%%%%%%%%%%%%%%%%%%%%%%%%%%%%%%%%%%%%%%%
% Annex.

% From now on, the chapters are referenciated as "Annex", and with an
% alpha-numeric identifier.
\annex


\chapter{Título do Anexo A}

bla bla bla bla bla bla bla bla bla bla bla bla

bla bla bla bla bla bla bla bla bla bla bla bla

%%%%%%%%%%%%%%%%%%%%%%%%%%%%%%%%%%%%%%%%%%%%%%%%%%%%%%%%%%%%%%%%%%%%%%

\chapter{Título do Anexo B}

bla bla bla bla bla bla bla bla bla bla bla bla

bla bla bla bla bla bla bla bla bla bla bla bla

%%%%%%%%%%%%%%%%%%%%%%%%%%%%%%%%%%%%%%%%%%%%%%%%%%%%%%%%%%%%%%%%%%%%%%

\end{document}
